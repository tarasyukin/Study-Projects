@article{lee_2022,
title = {Evidence to guide ethical decision-making in the management of older people living in squalor: a narrative review.},
author = {Lee, Sook Meng and Martino, Erika and Bismark, Marie and Bentley, Rebecca},
pages = {1304-1312},
url = {http://dx.doi.org/10.1111/imj.15862},
year = {2022},
month = {aug},
urldate = {2022-09-12},
journal = {Internal Medicine Journal},
volume = {52},
number = {8},
doi = {10.1111/imj.15862},
pmid = {35762169},
keywords = {geriatrics},
abstract = {Older people living in squalor present healthcare providers with a set of complex issues because squalor occurs alongside a variety of medical and psychiatric conditions, and older people living in squalor frequently decline intervention. To synthesise empirical evidence on squalor to inform ethical decision-making in the management of squalor using the bioethical framework of principlism. A systematic literature search was conducted using Medline, Embase, {PsycINFO} and {CINAHL} databases for empirical research on squalor in older people. Given the limited evidence base to date, an interpretive approach to synthesis was used. Sixty-seven articles that met the inclusion criteria were included in the review. Our synthesis of the research evidence indicates that: (i) older people living in squalor have a high prevalence of frontal executive dysfunction, medical comorbidities and premature deaths; (ii) interventions are complex and require interagency involvement, with further evaluations needed to determine the effectiveness and potential harm of interventions; and (iii) older people living in squalor utilise more medical and social resources, and may negatively impact others around them. These results suggest that autonomous decision-making capacity should be determined rather than assumed. The harm associated with squalid living for the older person, and for others around them, means a non-interventional approach is likely to contravene the principles of non-maleficence, beneficence and justice. Adequate assessment of decision-making capacity is of particular importance. To be ethical, any intervention undertaken must balance benefits, harms, resource utilisation and impact on others. \copyright 2022 The Authors. Internal Medicine Journal published by John Wiley \& Sons Australia, Ltd on behalf of Royal Australasian College of Physicians.}
}
